\section{Introduction}% ===> this file was generated automatically by noweave --- better not edit it
I have received a lot of questions asking, what does
Emacsy\footnote{Kickstarter page \url{http://kck.st/IY0Bau}} actually do?  What
restrictions does it impose on the GUI toolkit?  How is it possible to
not use any Emacs code? I thought it might be best if I were to
provide a minimal example program, so that people can see code that
illustrates Emacsy API usage.  

\section{Embedders' API}

Here are a few function prototypes defined in \verb|emacsy.h|.

\nwfilename{hello-emacsy.nw}\nwbegincode{1}\sublabel{NW1sLXkd-5kDCg-1}\nwmargintag{{\nwtagstyle{}\subpageref{NW1sLXkd-5kDCg-1}}}\moddef{emacsy.h~{\nwtagstyle{}\subpageref{NW1sLXkd-5kDCg-1}}}\endmoddef\nwstartdeflinemarkup\nwenddeflinemarkup
/* Initialize Emacsy. */
int  emacsy_initialize(void);

/* Enqueue a keyboard event. */
void emacsy_key_event(int char_code,
                      int modifier_key_flags);

/* Run an iteration of Emacsy's event loop 
   (will not block). */
int emacsy_tick(); 

/* Return the message or echo area. */
char *emacsy_message_or_echo_area();

/* Return the mode line. */
char *emacsy_mode_line();

/* Terminate Emacsy, runs termination hook. */
int  emacsy_terminate();
\nwnotused{emacsy.h}\nwendcode{}\nwbegindocs{2}\nwdocspar

\begin{figure} 
  \centering
  \includegraphics[scale=0.4]{minimal-emacsy-figure} 
  \caption[Short Label]{\label{minimal-emacsy-figure}Emacsy
    integrated into the simplest application ever!}
\end{figure} 

\section{The Simplest Application Ever}
 
Let's exercise these functions in a minimal GLUT program we'll call
\verb|hello-emacsy|.\footnote{Note: Emacsy does not rely on GLUT. One
  could use Qt, Cocoa, or ncurses.}  This simple program will display
an integer, the variable {\Tt{}counter\nwendquote}, that one can increment or
decrement.  The code will be organized as follows.

\nwenddocs{}\nwbegincode{3}\sublabel{NW1sLXkd-4LUyNL-1}\nwmargintag{{\nwtagstyle{}\subpageref{NW1sLXkd-4LUyNL-1}}}\moddef{file:hello-emacsy.c~{\nwtagstyle{}\subpageref{NW1sLXkd-4LUyNL-1}}}\endmoddef\nwstartdeflinemarkup\nwenddeflinemarkup
\LA{}Headers~{\nwtagstyle{}\subpageref{NW1sLXkd-2BmRoZ-1}}\RA{}
\LA{}State~{\nwtagstyle{}\subpageref{NW1sLXkd-1rXFj0-1}}\RA{}
\LA{}Functions~{\nwtagstyle{}\subpageref{NW1sLXkd-1byJZg-1}}\RA{}
\LA{}Primitives~{\nwtagstyle{}\subpageref{NW1sLXkd-3HbdVZ-1}}\RA{}
\LA{}Register primitives.~{\nwtagstyle{}\subpageref{NW1sLXkd-1OeWCX-1}}\RA{}
\LA{}Main~{\nwtagstyle{}\subpageref{NW1sLXkd-1uPnon-1}}\RA{}
\nwnotused{file:hello-emacsy.c}\nwendcode{}\nwbegindocs{4}\nwdocspar

Our application's state is captured by one global variable.

\nwenddocs{}\nwbegincode{5}\sublabel{NW1sLXkd-1rXFj0-1}\nwmargintag{{\nwtagstyle{}\subpageref{NW1sLXkd-1rXFj0-1}}}\moddef{State~{\nwtagstyle{}\subpageref{NW1sLXkd-1rXFj0-1}}}\endmoddef\nwstartdeflinemarkup\nwusesondefline{\\{NW1sLXkd-4LUyNL-1}}\nwenddeflinemarkup
int counter = 0; /* We display this number. */
\nwused{\\{NW1sLXkd-4LUyNL-1}}\nwendcode{}\nwbegindocs{6}\nwdocspar

Let's initialize everything in {\Tt{}main\nwendquote} and enter our runloop.

\nwenddocs{}\nwbegincode{7}\sublabel{NW1sLXkd-1uPnon-1}\nwmargintag{{\nwtagstyle{}\subpageref{NW1sLXkd-1uPnon-1}}}\moddef{Main~{\nwtagstyle{}\subpageref{NW1sLXkd-1uPnon-1}}}\endmoddef\nwstartdeflinemarkup\nwusesondefline{\\{NW1sLXkd-4LUyNL-1}}\nwenddeflinemarkup
int main(int argc, char *argv[]) \{
  int err;
  \LA{}Initialize GLUT.~{\nwtagstyle{}\subpageref{NW1sLXkd-2zNKGr-1}}\RA{}
  scm_init_guile();    /* Initialize Guile. */
  /* Initialize Emacsy. */
  err = emacsy_initialize(); 
  if (err)
    exit(err);
  primitives_init();   /* Register primitives. */
  \LA{}Load config.~{\nwtagstyle{}\subpageref{NW1sLXkd-2Z9J3a-1}}\RA{}
  glutMainLoop();      /* We never return. */
  return 0; 
\}
\nwused{\\{NW1sLXkd-4LUyNL-1}}\nwendcode{}\nwbegindocs{8}\nwdocspar


\section{Runloop Interaction}

Let's look at how Emacsy interacts with your application's runloop
since that's probably the most concerning part of embedding.  First,
let's pass some input to Emacsy.

\nwenddocs{}\nwbegincode{9}\sublabel{NW1sLXkd-1byJZg-1}\nwmargintag{{\nwtagstyle{}\subpageref{NW1sLXkd-1byJZg-1}}}\moddef{Functions~{\nwtagstyle{}\subpageref{NW1sLXkd-1byJZg-1}}}\endmoddef\nwstartdeflinemarkup\nwusesondefline{\\{NW1sLXkd-4LUyNL-1}}\nwprevnextdefs{\relax}{NW1sLXkd-1byJZg-2}\nwenddeflinemarkup
void keyboard_func(unsigned char glut_key, 
                   int x, int y) \{
  /* Send the key event to Emacsy 
     (not processed yet). */
  int key;
  int mod_flags;
  \LA{}Get modifier key flags.~{\nwtagstyle{}\subpageref{NW1sLXkd-2IySsV-1}}\RA{}
  \LA{}Handle control modifier.~{\nwtagstyle{}\subpageref{NW1sLXkd-GK1NJ-1}}\RA{}
  emacsy_key_event(key, 
                   mod_flags);
  glutPostRedisplay();
\}
\nwalsodefined{\\{NW1sLXkd-1byJZg-2}\\{NW1sLXkd-1byJZg-3}}\nwused{\\{NW1sLXkd-4LUyNL-1}}\nwendcode{}\nwbegindocs{10}The keys \verb|C-a| and \verb|C-b| returns $1$ and $2$
respectively. We want to map these to their actual character values.
\nwenddocs{}\nwbegincode{11}\sublabel{NW1sLXkd-GK1NJ-1}\nwmargintag{{\nwtagstyle{}\subpageref{NW1sLXkd-GK1NJ-1}}}\moddef{Handle control modifier.~{\nwtagstyle{}\subpageref{NW1sLXkd-GK1NJ-1}}}\endmoddef\nwstartdeflinemarkup\nwusesondefline{\\{NW1sLXkd-1byJZg-1}}\nwenddeflinemarkup
key = mod_flags & EY_MODKEY_CONTROL 
  ? glut_key + ('a' - 1) 
  : glut_key;
\nwused{\\{NW1sLXkd-1byJZg-1}}\nwendcode{}\nwbegindocs{12}\nwdocspar

The function {\Tt{}display{\_}func\nwendquote} is run for every frame that's
drawn. It's effectively our runloop, even though the actual runloop is
in GLUT.

\nwenddocs{}\nwbegincode{13}\sublabel{NW1sLXkd-1byJZg-2}\nwmargintag{{\nwtagstyle{}\subpageref{NW1sLXkd-1byJZg-2}}}\moddef{Functions~{\nwtagstyle{}\subpageref{NW1sLXkd-1byJZg-1}}}\plusendmoddef\nwstartdeflinemarkup\nwusesondefline{\\{NW1sLXkd-4LUyNL-1}}\nwprevnextdefs{NW1sLXkd-1byJZg-1}{NW1sLXkd-1byJZg-3}\nwenddeflinemarkup
/* GLUT display function */
void display_func() \{
  \LA{}Setup display.~{\nwtagstyle{}\subpageref{NW1sLXkd-26dPFc-1}}\RA{}
  \LA{}Display the counter variable.~{\nwtagstyle{}\subpageref{NW1sLXkd-11QRfI-1}}\RA{}

  /* Process events in Emacsy. */
  if (emacsy_tick() & EY_QUIT_APPLICATION) \{
    emacsy_terminate();
    exit(0);
  \}

  /* Display Emacsy message/echo area. */
  draw_string(0, 5, emacsy_message_or_echo_area());
  /* Display Emacsy mode line. */
  draw_string(0, 30, emacsy_mode_line());
        
  glutSwapBuffers();
\}
\nwused{\\{NW1sLXkd-4LUyNL-1}}\nwendcode{}\nwbegindocs{14}\nwdocspar
At this point, our application can process key events, accept input on
the minibuffer, and use nearly all of the facilities that Emacsy
offers, but it can't change any application state, which makes it not
very interesting yet.
\nwenddocs{}\nwbegindocs{15}\nwdocspar
\section{Plugging Into Your App}

Let's define a new primitive Scheme procedure {\Tt{}get-counter\nwendquote}, so
Emacsy can access the application's state.  This will define
a {\Tt{}C\nwendquote} function {\Tt{}SCM\ scm{\_}get{\_}counter(void)\nwendquote} and a Scheme procedure
{\Tt{}(get-counter)\nwendquote}.
\nwenddocs{}\nwbegincode{16}\sublabel{NW1sLXkd-3HbdVZ-1}\nwmargintag{{\nwtagstyle{}\subpageref{NW1sLXkd-3HbdVZ-1}}}\moddef{Primitives~{\nwtagstyle{}\subpageref{NW1sLXkd-3HbdVZ-1}}}\endmoddef\nwstartdeflinemarkup\nwusesondefline{\\{NW1sLXkd-4LUyNL-1}}\nwprevnextdefs{\relax}{NW1sLXkd-3HbdVZ-2}\nwenddeflinemarkup
SCM_DEFINE (scm_get_counter, "get-counter", 
            /* required arg count    */ 0,
            /* optional arg count    */ 0,
            /* variable length args? */ 0,
            (),
            "Returns value of counter.")
\{
  return scm_from_int(counter);
\}
\nwalsodefined{\\{NW1sLXkd-3HbdVZ-2}}\nwused{\\{NW1sLXkd-4LUyNL-1}}\nwendcode{}\nwbegindocs{17}Let's define another primitive Scheme procedure to alter the
application's state.
\nwenddocs{}\nwbegincode{18}\sublabel{NW1sLXkd-3HbdVZ-2}\nwmargintag{{\nwtagstyle{}\subpageref{NW1sLXkd-3HbdVZ-2}}}\moddef{Primitives~{\nwtagstyle{}\subpageref{NW1sLXkd-3HbdVZ-1}}}\plusendmoddef\nwstartdeflinemarkup\nwusesondefline{\\{NW1sLXkd-4LUyNL-1}}\nwprevnextdefs{NW1sLXkd-3HbdVZ-1}{\relax}\nwenddeflinemarkup
SCM_DEFINE (scm_set_counter_x, "set-counter!", 
         /* required, optional, var. length? */
            1, 0, 0, 
            (SCM value),
            "Sets value of counter.")
\{
  counter = scm_to_int(value);
  glutPostRedisplay();
  return SCM_UNSPECIFIED;
\}
\nwused{\\{NW1sLXkd-4LUyNL-1}}\nwendcode{}\nwbegindocs{19}Once we have written these primitive procedures, we need to register
them with the Scheme runtime.  

\nwenddocs{}\nwbegincode{20}\sublabel{NW1sLXkd-1OeWCX-1}\nwmargintag{{\nwtagstyle{}\subpageref{NW1sLXkd-1OeWCX-1}}}\moddef{Register primitives.~{\nwtagstyle{}\subpageref{NW1sLXkd-1OeWCX-1}}}\endmoddef\nwstartdeflinemarkup\nwusesondefline{\\{NW1sLXkd-4LUyNL-1}}\nwenddeflinemarkup
void primitives_init()
\{
#ifndef SCM_MAGIC_SNARFER
  #include "hello-emacsy.c.x"
#endif
\}
\nwused{\\{NW1sLXkd-4LUyNL-1}}\nwendcode{}\nwbegindocs{21}We generate the file \verb|hello-emacsy.x| by running the command:
\verb|guile-snarf hello-emacsy.c|. Emacsy can now access and alter the
application's internal state.

\section{Changing the UI}
Now let's use these new procedures to create interactive commands and
bind them to keys by changing our config file \verb|.hello-emacsy.scm|.

\nwenddocs{}\nwbegincode{22}\sublabel{NW1sLXkd-1kE3u6-1}\nwmargintag{{\nwtagstyle{}\subpageref{NW1sLXkd-1kE3u6-1}}}\moddef{file:.hello-emacsy.scm~{\nwtagstyle{}\subpageref{NW1sLXkd-1kE3u6-1}}}\endmoddef\nwstartdeflinemarkup\nwprevnextdefs{\relax}{NW1sLXkd-1kE3u6-2}\nwenddeflinemarkup
(use-modules (emacsy emacsy))

(define-interactive (incr-counter)
 (set-counter! (1+ (get-counter))))

(define-interactive (decr-counter)
 (set-counter! (1- (get-counter))))

(define-key global-map 
 (kbd "=") 'incr-counter)
(define-key global-map 
 (kbd "-") 'decr-counter)
\nwalsodefined{\\{NW1sLXkd-1kE3u6-2}\\{NW1sLXkd-1kE3u6-3}}\nwnotused{file:.hello-emacsy.scm}\nwendcode{}\nwbegindocs{23}\nwdocspar

We load this file in {\Tt{}main\nwendquote} like so.

\nwenddocs{}\nwbegincode{24}\sublabel{NW1sLXkd-2Z9J3a-1}\nwmargintag{{\nwtagstyle{}\subpageref{NW1sLXkd-2Z9J3a-1}}}\moddef{Load config.~{\nwtagstyle{}\subpageref{NW1sLXkd-2Z9J3a-1}}}\endmoddef\nwstartdeflinemarkup\nwusesondefline{\\{NW1sLXkd-1uPnon-1}}\nwenddeflinemarkup
if (access(".hello-emacsy.scm", R_OK) != -1) \{
  scm_c_primitive_load(".hello-emacsy.scm");
\} else \{
  fprintf(stderr, "warning: unable to load '.hello-emacsy.scm'.\\n");
\}
\nwused{\\{NW1sLXkd-1uPnon-1}}\nwendcode{}\nwbegindocs{25}\nwdocspar

We can now hit \verb|-| and \verb|=| to decrement and increment the
{\Tt{}counter\nwendquote}. This is fine, but what else can we do with it?  We could
make a macro that increments 5 times by hitting
\verb|C-x ( = = = = = C-x )|, then hit \verb|C-e| to run that macro.

Let's implement another command that will ask the user for a number to
set the counter to.

\nwenddocs{}\nwbegincode{26}\sublabel{NW1sLXkd-1kE3u6-2}\nwmargintag{{\nwtagstyle{}\subpageref{NW1sLXkd-1kE3u6-2}}}\moddef{file:.hello-emacsy.scm~{\nwtagstyle{}\subpageref{NW1sLXkd-1kE3u6-1}}}\plusendmoddef\nwstartdeflinemarkup\nwprevnextdefs{NW1sLXkd-1kE3u6-1}{NW1sLXkd-1kE3u6-3}\nwenddeflinemarkup
(define-interactive (change-counter) 
 (set-counter! 
   (string->number 
     (read-from-minibuffer 
       "New counter value: "))))
\nwendcode{}\nwbegindocs{27}\nwdocspar

Now we can hit \verb|M-x change-counter| and we'll be prompted for the
new value we want.  There we have it.  We have made the simplest
application ever more \emph{Emacs-y}.

\section{Changing it at Runtime}

We can add commands easily by changing and reloading the file.  But
we can do better.  Let's start a REPL we can connect to.

\nwenddocs{}\nwbegincode{28}\sublabel{NW1sLXkd-1kE3u6-3}\nwmargintag{{\nwtagstyle{}\subpageref{NW1sLXkd-1kE3u6-3}}}\moddef{file:.hello-emacsy.scm~{\nwtagstyle{}\subpageref{NW1sLXkd-1kE3u6-1}}}\plusendmoddef\nwstartdeflinemarkup\nwprevnextdefs{NW1sLXkd-1kE3u6-2}{\relax}\nwenddeflinemarkup
(use-modules (system repl server))

;; Start a server on port 37146.
(spawn-server)
\nwendcode{}\nwbegindocs{29}\nwdocspar

Now we can \verb|telnet localhost 37146| to get a REPL.  

\section{Conclusion}
We implemented a simple interactive application that displays a
number.  We embedded Emacsy into it: sending events to Emacsy and
displaying the minibuffer.  We implemented primitive procedures so
Emacsy could access and manipulate the application's state.  We
extended the user interface to accept new commands \verb|+| and
\verb|-| to change the state.

%\newpage
%\appendix
\begin{subappendices}

\section{Plaintext Please}
Here are the plaintext files: \href{http://gnufoo.org/emacsy/emacsy.h}{emacsy.h},
\href{http://gnufoo.org/emacsy/hello-emacsy.c}{hello-emacsy.c},
\href{http://gnufoo.org/emacsy/emacsy-stub.c}{emacsy-stub.c}, and
\href{http://gnufoo.org/emacsy/.hello-emacsy.scm}{.hello-emacsy.scm}. Or 

\section{Uninteresting Code}
Not particularly interesting bits of code but necessary to compile.

\lstset{basicstyle=\footnotesize}

\nwenddocs{}\nwbegincode{30}\sublabel{NW1sLXkd-2BmRoZ-1}\nwmargintag{{\nwtagstyle{}\subpageref{NW1sLXkd-2BmRoZ-1}}}\moddef{Headers~{\nwtagstyle{}\subpageref{NW1sLXkd-2BmRoZ-1}}}\endmoddef\nwstartdeflinemarkup\nwusesondefline{\\{NW1sLXkd-4LUyNL-1}}\nwenddeflinemarkup
#ifndef SCM_MAGIC_SNARFER
#ifdef __APPLE__
#include <GLUT/glut.h> 
#else
#include <GL/glut.h> 
#endif
#include <stdlib.h>
#include <emacsy.h>
#endif
#include <libguile.h>

void draw_string(int, int, char*);
\nwused{\\{NW1sLXkd-4LUyNL-1}}\nwendcode{}\nwbegindocs{31}\nwdocspar

%Draw a string function.
\nwenddocs{}\nwbegincode{32}\sublabel{NW1sLXkd-1byJZg-3}\nwmargintag{{\nwtagstyle{}\subpageref{NW1sLXkd-1byJZg-3}}}\moddef{Functions~{\nwtagstyle{}\subpageref{NW1sLXkd-1byJZg-1}}}\plusendmoddef\nwstartdeflinemarkup\nwusesondefline{\\{NW1sLXkd-4LUyNL-1}}\nwprevnextdefs{NW1sLXkd-1byJZg-2}{\relax}\nwenddeflinemarkup
/* Draws a string at (x, y) on the screen. */
void draw_string(int x, int y, char *string) \{
  glLoadIdentity();
  glTranslatef(x, y, 0.);
  glScalef(0.2, 0.2, 1.0);
  while(*string) 
    glutStrokeCharacter(GLUT_STROKE_ROMAN, 
                        *string++);
\}
\nwused{\\{NW1sLXkd-4LUyNL-1}}\nwendcode{}\nwbegindocs{33}\nwdocspar

Setup the display buffer the drawing.
\nwenddocs{}\nwbegincode{34}\sublabel{NW1sLXkd-26dPFc-1}\nwmargintag{{\nwtagstyle{}\subpageref{NW1sLXkd-26dPFc-1}}}\moddef{Setup display.~{\nwtagstyle{}\subpageref{NW1sLXkd-26dPFc-1}}}\endmoddef\nwstartdeflinemarkup\nwusesondefline{\\{NW1sLXkd-1byJZg-2}}\nwenddeflinemarkup
glClear(GL_COLOR_BUFFER_BIT);

glMatrixMode(GL_PROJECTION);
glLoadIdentity();
glOrtho(0.0, 500.0, 0.0, 500.0, -2.0, 500.0);
gluLookAt(0,   0,   2, 
          0.0, 0.0, 0.0, 
          0.0, 1.0, 0.0);

glMatrixMode(GL_MODELVIEW);
glColor3f(1, 1, 1);
\nwused{\\{NW1sLXkd-1byJZg-2}}\nwendcode{}\nwbegindocs{35}\nwdocspar

\nwenddocs{}\nwbegincode{36}\sublabel{NW1sLXkd-2zNKGr-1}\nwmargintag{{\nwtagstyle{}\subpageref{NW1sLXkd-2zNKGr-1}}}\moddef{Initialize GLUT.~{\nwtagstyle{}\subpageref{NW1sLXkd-2zNKGr-1}}}\endmoddef\nwstartdeflinemarkup\nwusesondefline{\\{NW1sLXkd-1uPnon-1}}\nwenddeflinemarkup
glutInit(&argc, argv);
glutInitDisplayMode(GLUT_RGB|GLUT_DOUBLE);
glutInitWindowSize(500, 500);
glutCreateWindow("Hello, Emacsy!");
glutDisplayFunc(display_func);
glutKeyboardFunc(keyboard_func);
\nwused{\\{NW1sLXkd-1uPnon-1}}\nwendcode{}\nwbegindocs{37}\nwdocspar

Our application has just one job.

\nwenddocs{}\nwbegincode{38}\sublabel{NW1sLXkd-11QRfI-1}\nwmargintag{{\nwtagstyle{}\subpageref{NW1sLXkd-11QRfI-1}}}\moddef{Display the counter variable.~{\nwtagstyle{}\subpageref{NW1sLXkd-11QRfI-1}}}\endmoddef\nwstartdeflinemarkup\nwusesondefline{\\{NW1sLXkd-1byJZg-2}}\nwenddeflinemarkup
char counter_string[255];
sprintf(counter_string, "%d", counter);
draw_string(250, 250, counter_string);
\nwused{\\{NW1sLXkd-1byJZg-2}}\nwendcode{}\nwbegindocs{39}\nwdocspar

\nwenddocs{}\nwbegincode{40}\sublabel{NW1sLXkd-2IySsV-1}\nwmargintag{{\nwtagstyle{}\subpageref{NW1sLXkd-2IySsV-1}}}\moddef{Get modifier key flags.~{\nwtagstyle{}\subpageref{NW1sLXkd-2IySsV-1}}}\endmoddef\nwstartdeflinemarkup\nwusesondefline{\\{NW1sLXkd-1byJZg-1}}\nwenddeflinemarkup
int glut_mod_flags = glutGetModifiers();
mod_flags = 0;
if (glut_mod_flags & GLUT_ACTIVE_SHIFT)
   mod_flags |= EY_MODKEY_SHIFT;
if (glut_mod_flags & GLUT_ACTIVE_CTRL)
   mod_flags |= EY_MODKEY_CONTROL;
if (glut_mod_flags & GLUT_ACTIVE_ALT)
   mod_flags |= EY_MODKEY_META;
\nwused{\\{NW1sLXkd-1byJZg-1}}\nwendcode{}

\nwixlogsorted{c}{{Display the counter variable.}{NW1sLXkd-11QRfI-1}{\nwixu{NW1sLXkd-1byJZg-2}\nwixd{NW1sLXkd-11QRfI-1}}}%
\nwixlogsorted{c}{{emacsy.h}{NW1sLXkd-5kDCg-1}{\nwixd{NW1sLXkd-5kDCg-1}}}%
\nwixlogsorted{c}{{file:.hello-emacsy.scm}{NW1sLXkd-1kE3u6-1}{\nwixd{NW1sLXkd-1kE3u6-1}\nwixd{NW1sLXkd-1kE3u6-2}\nwixd{NW1sLXkd-1kE3u6-3}}}%
\nwixlogsorted{c}{{file:hello-emacsy.c}{NW1sLXkd-4LUyNL-1}{\nwixd{NW1sLXkd-4LUyNL-1}}}%
\nwixlogsorted{c}{{Functions}{NW1sLXkd-1byJZg-1}{\nwixu{NW1sLXkd-4LUyNL-1}\nwixd{NW1sLXkd-1byJZg-1}\nwixd{NW1sLXkd-1byJZg-2}\nwixd{NW1sLXkd-1byJZg-3}}}%
\nwixlogsorted{c}{{Get modifier key flags.}{NW1sLXkd-2IySsV-1}{\nwixu{NW1sLXkd-1byJZg-1}\nwixd{NW1sLXkd-2IySsV-1}}}%
\nwixlogsorted{c}{{Handle control modifier.}{NW1sLXkd-GK1NJ-1}{\nwixu{NW1sLXkd-1byJZg-1}\nwixd{NW1sLXkd-GK1NJ-1}}}%
\nwixlogsorted{c}{{Headers}{NW1sLXkd-2BmRoZ-1}{\nwixu{NW1sLXkd-4LUyNL-1}\nwixd{NW1sLXkd-2BmRoZ-1}}}%
\nwixlogsorted{c}{{Initialize GLUT.}{NW1sLXkd-2zNKGr-1}{\nwixu{NW1sLXkd-1uPnon-1}\nwixd{NW1sLXkd-2zNKGr-1}}}%
\nwixlogsorted{c}{{Load config.}{NW1sLXkd-2Z9J3a-1}{\nwixu{NW1sLXkd-1uPnon-1}\nwixd{NW1sLXkd-2Z9J3a-1}}}%
\nwixlogsorted{c}{{Main}{NW1sLXkd-1uPnon-1}{\nwixu{NW1sLXkd-4LUyNL-1}\nwixd{NW1sLXkd-1uPnon-1}}}%
\nwixlogsorted{c}{{Primitives}{NW1sLXkd-3HbdVZ-1}{\nwixu{NW1sLXkd-4LUyNL-1}\nwixd{NW1sLXkd-3HbdVZ-1}\nwixd{NW1sLXkd-3HbdVZ-2}}}%
\nwixlogsorted{c}{{Register primitives.}{NW1sLXkd-1OeWCX-1}{\nwixu{NW1sLXkd-4LUyNL-1}\nwixd{NW1sLXkd-1OeWCX-1}}}%
\nwixlogsorted{c}{{Setup display.}{NW1sLXkd-26dPFc-1}{\nwixu{NW1sLXkd-1byJZg-2}\nwixd{NW1sLXkd-26dPFc-1}}}%
\nwixlogsorted{c}{{State}{NW1sLXkd-1rXFj0-1}{\nwixu{NW1sLXkd-4LUyNL-1}\nwixd{NW1sLXkd-1rXFj0-1}}}%
\nwbegindocs{41}\nwdocspar
\end{subappendices}
\nwenddocs{}
